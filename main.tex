%%%%%%%%%%%%%%%%%%%%%%%%%%%%%%%%%%%%%%%%%%%%%%%%%%%%
% Template para apresentação
% Versão SIMOT (15/11/2024)
%%%%%%%%%%%%%%%%%%%%%%%%%%%%%%%%%%%%%%%%%%%%%%%%%%%%

\documentclass[aspectratio=169]{beamer}

\setbeamertemplate{navigation symbols}{}

\mode<presentation> {

\usetheme{Madrid}

}

\usepackage{graphicx}
\usepackage{booktabs} % Uso de \toprule, \midrule e \bottomrule em tabelas.
\usepackage[brazil]{babel}
\usepackage[utf8]{inputenc}
\usepackage[T1]{fontenc}
\usepackage{amsfonts, amsmath, amssymb}
\usepackage{multirow}
\usepackage[brazilian,abnt-emphasize=bf,abnt-etal-list=0,abnt-etal-text=it]{abntex2cite}

%------------------------------------------------
% Primeiro Slide
\title[SIMOT]{Título}

\author{Nome do autor}

\institute[]{
UNIVERSIDADE FEDERAL DE CATALÃO \\
UNIDADE ACADÊMICA DE MATEMÁTICA E TECNOLOGIA \\
PROGRAMA DE PÓS-GRADUAÇÃO EM MODELAGEM E OTIMIZAÇÃO \\
\medskip
\textit{Orientador: }
}
\date{}

\begin{document}

\begin{frame}
\titlepage
\end{frame}
%------------------------------------------------
\logo{\includegraphics[height=1.5cm]{figuras/SIMOT.png}}
%------------------------------------------------

\begin{frame}
\frametitle{Introdução}

\begin{block}{Discentes: 1º ou 2º Semestre \cite{teste}}
    \begin{itemize}
        \item Fazer uma contextualização. Caso o tema ainda não esteja definido, descrever sobre as atividades já realizadas durante o período do mestrado e sobre possíveis temas que podem ser escolhidos;
        \item Descrever os objetivos. Caso o tema ainda não esteja definido, descrever sobre as motivações em relação aos temas mencionados.
    \end{itemize}
\end{block}

\begin{block}{Discentes: 3º ou 4º Semestre \cite{teste}}
    \begin{itemize}
        \item Fazer uma contextualização;
        \item Descrever os objetivos.
    \end{itemize}
\end{block}

\end{frame}
%------------------------------------------------

\begin{frame}
\frametitle{Metodologia}

\begin{block}{Discentes: 1º ou 2º Semestre \cite{teste}}
    \begin{itemize}
        \item Descrever os passos que já foram tomados;
    \end{itemize}
\end{block}

\begin{block}{Discentes: 3º ou 4º Semestre \cite{teste}}
    \begin{itemize}
        \item Descrever sobre as técnicas que serão utilizadas;
        \begin{itemize}
            \item Quais foram utilizadas?
            \item Como foram utilizadas?
        \end{itemize}
    \end{itemize}
\end{block}

\end{frame}
%------------------------------------------------

\begin{frame}
\frametitle{Resultados}

\begin{block}{Discentes: 1º ou 2º Semestre \cite{teste}}
    \begin{itemize}
        \item Apresentação de resultados não é obrigatória;
        \item Apresente o cronograma de atividades.
        \begin{itemize}
            \item Neste caso, não é necessário a existência dos slides "Resultados" e "Considerações finais";
            \item Delete estes dois e crie um slide intitulado "Cronograma de atividades".
        \end{itemize}
    \end{itemize}
\end{block}

\begin{block}{Discentes: 3º ou 4º Semestre \cite{teste}}
    \begin{itemize}
        \item Esta etapa de apresentação dos resultados é obrigatória, mesmo que sejam resultados parciais;
    \end{itemize}
\end{block}

\end{frame}
%------------------------------------------------

\begin{frame}
\frametitle{Considerações finais}

\begin{block}{Discentes: 1º ou 2º Semestre \cite{teste}}
    \begin{itemize}
        \item Caso não tenha resultados, este slide pode ser ignorado.
    \end{itemize}
\end{block}

\begin{block}{Discentes: 3º ou 4º Semestre \cite{teste}}
    \begin{itemize}
        \item Descrever o que foi observado nos resultados e o que falta a ser realizado;
        \item Apresentar o cronograma de atividades.
    \end{itemize}
\end{block}

\end{frame}
%------------------------------------------------

\begin{frame}[allowframebreaks]{Referências}
	\bibliography{referencias}
\end{frame}

\end{document}

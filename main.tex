%%%%%%%%%%%%%%%%%%%%%%%%%%%%%%%%%%%%%%%%%%%%%%%%%%%%
% Template para apresentação
% Versão SIMOT (15/11/2024)
%%%%%%%%%%%%%%%%%%%%%%%%%%%%%%%%%%%%%%%%%%%%%%%%%%%%

\documentclass[aspectratio=169]{beamer}

\setbeamertemplate{navigation symbols}{}

\mode<presentation> {

\usetheme{Madrid}

}

\usepackage{graphicx}
\usepackage{booktabs} % Uso de \toprule, \midrule e \bottomrule em tabelas.
\usepackage[brazil]{babel}
\usepackage[utf8]{inputenc}
\usepackage[T1]{fontenc}
\usepackage{amsfonts, amsmath, amssymb}
\usepackage{multirow}

%------------------------------------------------
% Primeiro Slide
\title[SIMOT]{Titulo}

\author{Nome do autor}

\institute[]{
UNIVERSIDADE FEDERAL DE CATALÃO \\
UNIDADE ACADÊMICA DE MATEMÁTICA E TECNOLOGIA \\
PROGRAMA DE PÓS-GRADUAÇÃO EM MODELAGEM E OTIMIZAÇÃO \\
\medskip
\textit{Orientador: }
}
\date{}

\begin{document}

\begin{frame}
\titlepage
\end{frame}
%------------------------------------------------
\logo{\includegraphics[height=1.5cm]{figuras/SIMOT.png}}
%------------------------------------------------

\begin{frame}
\frametitle{Contextualização}

\begin{block}{Novatos: 1º ou 2º Semestre}
    \begin{itemize}
        \item Fazer uma contextualização sobre o tema escolhido;
        \begin{itemize}
            \item Caso não tenha, descrever possíveis temas a serem realizados.
        \end{itemize}
        \item Motivação sobre o tema escolhido ou que ainda será escolhido.
    \end{itemize}
\end{block}

\begin{block}{Veteranos: 3º ou 4º Semestre}
    \begin{itemize}
        \item Contextualização sobre o tema escolhido;
        \item Motivação sobre o tema escolhido.
    \end{itemize}
\end{block}

\end{frame}
%------------------------------------------------

\begin{frame}
\frametitle{Metodologia}

\begin{block}{Novatos: 1º ou 2º Semestre}
    \begin{itemize}
        \item Descrever os passos que já foram tomados;
    \end{itemize}
\end{block}

\begin{block}{Veteranos: 3º ou 4º Semestre}
    \begin{itemize}
        \item Descrever os passos que já foram tomados;
        \item Descrever sobre a aquisição dos dados;
        \begin{itemize}
            \item Onde foram obtidos?
            \item Como foram obtidos?
        \end{itemize}
        \item Descrever sobre as técnicas que serão utilizadas;
        \begin{itemize}
            \item Quais serão utilizadas?
            \item Como serão utilizadas?
        \end{itemize}
    \end{itemize}
\end{block}

\end{frame}
%------------------------------------------------

\begin{frame}
\frametitle{Resultados}

\begin{block}{Novatos: 1º ou 2º Semestre}
    \begin{itemize}
        \item Apresentação de resultados não é obrigatória; 
        \item Caso não tenha resultados parciais, apresente o cronograma de atividades.
        \begin{itemize}
            \item Neste caso, não é necessário a existência dos slides "Resultados" e "Considerações finais";
            \item Delete estes dois e crie uma slide intitulado "Cronograma de atividades".
        \end{itemize}
    \end{itemize}
\end{block}

\begin{block}{Veteranos: 3º ou 4º Semestre}
    \begin{itemize}
        \item Esta etapa de apresentação dos resultados é obrigatória, mesmo que sejam resultados parciais;
    \end{itemize}
\end{block}

\end{frame}
%------------------------------------------------

\begin{frame}
\frametitle{Considerações finais}

\begin{block}{Novatos: 1º ou 2º Semestre}
    \begin{itemize}
        \item Caso não tenha resultados, este slide pode ser ignorado.
    \end{itemize}
\end{block}

\begin{block}{Veteranos: 3º ou 4º Semestre}
    \begin{itemize}
        \item Descrever o que foi observado nos resultados e o que falta a ser realizado;
        \item Apresentar o cronograma de atividades.
    \end{itemize}
\end{block}

\end{frame}
%------------------------------------------------

\end{document}
